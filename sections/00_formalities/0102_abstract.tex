\documentclass[../../main.tex]{subfiles}
\begin{document}

\thispagestyle{empty}
\vspace{0pt}
\begin{abstract}
\hspace{-3pt}
As online customers increasingly demand one- and same-day shipping, the inventory location is imperative; laws of physics requires inventory to be close to the customers. In a supply chain network of multiple warehouses that covers a large geographical area of customers, a warehouse may face stock-out while another warehouse has excess inventory.

\textbf{What is the problem?}\\
A stock-out in a warehouse will increase the distance to the customers located around this warehouse.

\textbf{Why is it important?}\\
In order to deliver within a day or less, inventory needs to be close to the customer.

\textbf{What is the solution?}\\
By using proactive transshipments from a warehouse with excess inventory to a warehouse facing stock-out, the distance to customers is reduced, which is likely to increase delivery speeds.

\textbf{How are you contributing to the literature?}\\
There is little research available on stock-out prediction in supply chain, and none using real data. This thesis aims to close this gap by applying available theory on a real-life example.

There is no literature on using machine learning in a lateral transshipment policy. This thesis adds to the existing literature on lateral transshipments by using machine learning in the policy.

\textbf{What is the result?}\\
Using a snapshot of the supply chain, a machine learning classifier predicts whether a warehouse will face stock-out in the next week. If yes, a machine learning model will predict additional units needed to avoid a stock-out. These predictions are used in a policy for lateral transshipments. If a warehouse has excess inventory, it can initiate a lateral transshipment to a warehouse which faces stock-out. As a result of the rebalancing, distance to customers is likely to reduce, which facilitates fast delivery offered by the retailer.

\end{abstract}


\end{document}