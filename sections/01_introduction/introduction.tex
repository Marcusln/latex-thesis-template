\documentclass[../../main.tex]{subfiles}
\begin{document}
% START SKRIV HER

\chapter{Introduction}

https://www.mckinsey.com/industries/retail/our-insights/same-day-delivery-ready-for-takeoff

https://www.wsj.com/articles/the-prime-effect-how-amazons-2-day-shipping-is-disrupting-retail-1537448425

It is all about e-commerce now. Online sales has increased at a 1\% compound annual growth rate from 2016-2020 (McKinsey, 2020), and traditional retailers are forced to follow customers to the Internet (MIT, 2013). Consumers enjoy the wide selection, low prices, customer reviews and the overall convenience online retail offers (HBR, 2011). In recent years, increasing competition in e-commerce has left consumers demanding more – they also want their orders to be delivered quickly. Traditionally, online retailers could achieve cost advantages by leveraging economies of scale at large warehouses in low-cost areas (MIT, 2018). However, as online retailers such as Amazon has popularized one-day shipping, and even same-day shipping, this strategy is no longer viable. Inventory needs to be in proximity to  customers, and advanced analytics is the only way to achive this without dramatically increasing costs (ibid).

In general, retailers tend to have lower profit margins compared to other sectors (Stern, 2021). The high elasticity of demand translates to fierce competition and a need for operational excellence. In other words, supply chains compete, not companies (Christopher, 1992). The highly dynamic and volatile nature of supply chain is a great challenge, characterized by conflicting goals. On the one hand, the goal is to minimize a supply chain's costs. On the other hand, the goal is to maximize customer service level. A key part of this trade-off lies in inventory optimization. Holding inventory is expensive, so it is beneficial to hold as little as possible. However, if the retailer does not have sufficient inventory to cover demand, customers are likely to go to competitors. This is particularly true for online retailers, where the barrier to choose a competitor is extremely low. 

Retailers purchase inventory based on forecasted demand. Since actual demand will deviate from the forecast, a buffer of safety stock is required to mitigate the risk of stockouts due to uncertainty in supply and demand. In a supply chain network with multiple demand regions, a typical problem is that warehouses in some regions are overstocked while others are understocked. In these cases, a *lateral transshipment* from the overstocked warehouse to the understocked warehouse will reduce the risk of stockout without increasing the overall inventory in the network. That is, by using lateral transshipments, the same service level can be achieved with a lower safety stock  (Ge, 2019). Finally, if the understocked warehouse is the only one that can enable fast delivery to customers in the region, avoiding stockouts is essential for maintaining fast shipping speeds.

This paper will present a model that predicts stockouts in a warehouse. The prediction is then used in a policy for lateral transshipments, where one warehouse can send excess inventory to the warehouse at risk of being out of stock.




% SLUTT SKRIV HER
\end{document}