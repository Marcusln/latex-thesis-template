\documentclass[../../main.tex]{subfiles}
\begin{document}
% START SKRIV HER

\section{Inventory Theory}

\begin{itemize}
\item EOQ model
\item Newsvendor problem
\item Safety stock
\end{itemize}


The inventory management problem is a trade-off between service-level and cost. Holding inventory is costly, so why hold inventory at all? Firstly, assuming lead time between the purchasing order and delivery from the vendor, the retailer needs to hold inventory to satisfy demand during lead time. Secondly, since customer demand is hard to predict, we need to hold inventory to cover the forecasted demand, plus additional safety stock to protect against demand uncertainty. Safety stock is … In general, the uncertainty the supply chain faces in terms of customer demand and vendor lead time. Finally, by ordering large quantities, economies of scale are achieved since the vendors likely offer bulk discounts and transportation is generally cheaper for large quantities. 

Definition of safety stock.

The now-classic Economic Lot Size Model introduced by Ford Harris (1913/15) showed… Even though the assumptions used is unrealistic, it is considered essential still to this day (Snyder, 2008).

% SLUTT SKRIV HER
\end{document}