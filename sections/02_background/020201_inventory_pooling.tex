\documentclass[../../main.tex]{subfiles}
\begin{document}
% START SKRIV HER

\section{Inventory Pooling}

Risk pooling is an important concept in supply chain management, as it can reduce the need for safety stock. Conceptionally, aggregated demand across locations will reduce demand variability since higher-than-expected demand in one location is likely to be offset by lower-than-expected demand in another location. Similarly, aggregated forecasts are more accurate than…   hence reduce the risk and need for safety stock.

A strategy for risk pooling could be to centralize inventory in one central warehouse. In our case, instead of one warehouse each in Bergen and Oslo, we could have one central warehouse in between, which would serve both Bergen and Oslo customers. Demand for Bergen/Oslo area would then be aggregated, and demand variability would be reduced. Since safety stock is the stock to cover demand variability, this would be reduced, hence the average needed inventory position needed will also reduce.

The benefit of risk pooling increases as demand variability increases. An increase in the coefficient of variation translates to an increased need for safety stock. If two demand regions are aggregated, however, the coefficient of variation would reduce, hence reducing the need for safety stock and also the average inventory needed. This would drive down cost of storing inventory, or it can free up resources for new inventory, which will increase the selection for customers.

The benefit of risk pooling comes from the fact that increased demand in one region is offset by decreased demand in another region. Consequently, the benefit is reduced when demand in different locations becomes more positively correlated. 

Due to the risk pooling effect, retailers might favor centralized over decentralized systems, i.e. a central warehouses over having one warehouse in each demand region. In terms of safety stock, service level and fixed costs, it will be beneficial. As demand variability decreases, safety stock can be reduced. When the central warehouse has the same inventory as the regional warehouses combined, and demand for the regions are not perfectly correlated, the service level will be higher for the centralized system. Finally, fixed costs is lower with one warehouse compared to multiple.

Definition of service level.

However, centralized systems are not always favorable over decentralized ones. First, since the distance to customers will be larger, customer lead times are likely to be much higher. Since transportation costs are a function of distance, these are also likely to be higher. In the online retail industry, customer lead time is a major competitive advantage (source). In our case, it is not possible to deliver same-day delivery using a centralized warehouse. Since this is a major concern for our company, a centralized system is not an option.

•	A table with centralized vs decentralized might be useful here

Trade-off inventory levels

*	High safety stock costly
*	Low safety stock risky as it increases the chance of lost sales (especially in online retail)
*	First rule of inventory management: Forecast is always wrong
*	Second rule: Aggregate demand information is always more accurate than dis-aggregate data
*	=> Combined: Risk pooling

% SLUTT SKRIV HER
\end{document}