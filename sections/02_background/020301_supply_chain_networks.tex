\documentclass[../../main.tex]{subfiles}
\begin{document}
% START SKRIV HER

\section{Supply Chain Networks}

*	Illustration of network
*	Single-echelon, multi-location, continuous review (s,S), with partial pooling.*
*	Få frem hvorfor vi vil være nærme kunden
*	Two warehouses, Oslo and Bergen. They serve Norway’s national demand, but only the warehouse in Bergen can fulfill to customers the same day. However, we can transfer inventory from Oslo to Bergen, and then fulfill to customers within one day. Receives inventory from vendors. Retail. Continuous order policy (s,S). 
*	Assumption: Oslo+Bergen has all the inventory customer demands, hence only difference is that Oslo warehouse can fulfill same-day to Oslo customers.

http://article.sapub.org/10.5923.j.logistics.20200901.01.html:
According to Hwarng & Xie (2008) SCs are very difficult to manage as they are dynamic and highly characterised by the bullwhip effect which is greatly influenced by vast changes in consumer demand. They add that simulation and modelling tools are prerequisite in gaining resilience in these SCs. Hou (2013) describe simulation as a process of experimenting models in a controlled virtual environment so as to replicate the working of a system in order to understand the system better. Simulation assists in designing and evaluating performance of complex SC facilities before actual implementation which allows better utilization of resources and also analyse the existing facilities to identify the extent to which SC problems are in order to modify them (Chan & Smith, 1993).

https://www.researchgate.net/publication/221529831_The_Value_of_Simulation_in_Modeling_Supply_Chains:



% SLUTT SKRIV HER
\end{document}