\documentclass[../../main.tex]{subfiles}
\begin{document}
% START SKRIV HER

\newcommand{\headerline}[3]{%
  \par\medskip\noindent
  \makebox[0pt][l]{#1}%
  \makebox[\textwidth][c]{#2}%
  \makebox[0pt][r]{\texttt{#3}}\par\medskip}

\iffalse
\hspace*{\fill}
{ILLUSTRATION 1} \hfill {ILLUSTRATION 2}
\hspace*{\fill}
\fi

\iffalse
\begin{enumerate}[a)]
    \item \hfill$\begin{aligned}[t]
        &h_i + p_j - c_{ij} - (c_i - c_j) \geq 0
    \end{aligned}$\hfill\null
    \item \hfill$\begin{aligned}[t]
        &c_{ij} + (c_i - c_j) - (h_i - h_j) \geq 0
    \end{aligned}$\hfill\null
    \item \hfill$\begin{aligned}[t]
        &c_{ij} + (c_i - c_j) - (p_i - p_j) \geq 0
    \end{aligned}$\hfill\null
\end{enumerate}
\fi

\section{Lateral Transshipments}

\begin{itemize}
\item definition
\item Paterson review
\item Proactive vs reactive
\item adds complexity -> simplication
\item but makes SC more flexible
\item key characteristics table
\item inventory pooling (complete, partial)
\end{itemize}

\begin{enumerate}[a)]
\item this is item a
\item another item
\end{enumerate}

Lateral transshipments are shipments between nodes of the same echelon in the supply chain. Typically, this means that a warehouse with excess inventory can re-supply another warehouse which requires additional inventory. Since a warehouse's excess inventory is leveraged to avoid lost sales or backorders due to stock-out, they are a mechanism to increase the service level. In effect, multiple inventory locations are consolidated, and the inventory pooling effect is achieved, which reduces costs by reducing the safety stock needed. By using transshipments, one can achieve the benefits of risk pooling without a central warehouse. 

A key principle in forecasting is that forecasts are less accurate for longer time horizons (Reid, 2005@). Retailers places orders based on demand forecasts. From the retailer places the order until the demand is realized. The actual realized demand can be completely different than the forecasted demand at the time the retailer places the order. Due to these differences, it might happen that the inventory positions between the warehouses get imbalanced. A lateral transshipment can transfer stock to rebalance the network. 

Similar to risk pooling => Since lateral transshipments leverage one warehouse's excess inventory to reduce another warehouse's shortfall, they are a mechanism to increase the service level.

Increase service level or reduce cost, reduce inventory level.

Compared to placing a new purchasing order, a lateral transshipment usually has a much less lead time, making it an attractive option when demand must be met instantaneously. This is the case for an online retailer when it needs to calculate the offered delivery time when a customer searches for ... However, it costs at the cost of transshipment, which is likely to be higher than replenishment from a central warehouse, as they are smaller and lack economies of scale. (Snyder 2019@)

In a case where the order lead time is sufficiently large, a lateral transshipment can .. of get inventory without having to order. However, this is more costly as they are usually smaller and hence less scalable. Problem: How to scale lateral transshipments. Gather up transfer suggestions to fill up a full truck.

In a comprehensive review, Paterson (2011@) classified literature by key characteristics related to inventory system, ordering and type of transshipment. A key characteristic is between proactive transshipments that occur at fixed points in time, and reactive transshipments that can occur at any time. Reactive transshipments are made in \textit{reaction} to realized demand, while proactive transshipments are made in anticipation of demand shortages. 

Complexity increases if ordering decisions are considered in the transshipment policy. Thus, they are either ignored completely or very restrictive assumptions are made (Paterson, 2011@). 

Since the policy proposed in this thesis ignores ordering, we omit these details from the literature review.

if benefit of shipping cost is greater than transfer cost

Literature is usually about retailers sharing inventory with each other. 

Tagaras (1989@) serves as an accessible illustration of a reactive transshipment policy. The considered supply chain consists of two retailers that are served by a single central warehouse and are allowed to transship to each other. Inventory levels are reviewed every period and each retailer follows a base-stock policy with base-stock \(S_i\). After demand $D_i$ is realized, transshipment decisions are made before demand is satisfied. We observe the following costs:
%
\begin{align*}
    &c_i = \text{ordering cost per unit at retailer $i$}, && \text{for $i = 1,2$}  \\ 
    &h_i = \text{holding cost per unit at retailer $i$}, && \text{for $i = 1,2$}  \\ 
    &p_i = \text{penalty cost for stock-out per unit at retailer $i$}, && \text{for $i = 1,2$}  \\ 
    &c_{ij} = \text{cost per unit to transship from $i$ to $j$}, && \text{$i,j = 1,2; i \neq j\,$.}
\end{align*}
%
The assumption
%
\begin{equation}
    c_i - c_j + c_{ij} \geq 0\label{eq:1}
\end{equation}
%
ensures that it is cheaper to ship directly to \(j\) than to ship to \(i\) and immediately transship to \(j\). Additionally, we make the following assumptions: 
%
\begin{align}
    h_i + p_j - c_{ij} - (c_i - c_j) &\geq 0\label{eq:2} \\
    c_{ij} + (c_i - c_j) - (h_i - h_j) &\geq 0\label{eq:3} \\
    c_{ij} + (c_i - c_j) - (p_i - p_j) &\geq 0\label{eq:4}\,.
\end{align}
%
Assumption \eqref{eq:2} shows that a transshipment from retailer \(i\) to \(j\) is beneficial, since the cost of not doing a transshipment is $h_i + p_j + c_j - c_i$, which is less than the cost of the transshipment. Assumption \eqref{eq:3} and \eqref{eq:4} ensures that no transshipment is done if both retailers are over-stocked or under-stocked, respectively. Combined, the assumptions imply that a complete pooling policy is optimal: Transship $Y_{ij}$ from retailer $i$ to $j$, where $Y_{ij}$ is either the surplus stock of $i$, or only the part of the surplus stock that $j$ requires to satisfy demand. Formally, the complete pooling policy can be expressed as
% 
\begin{enumerate}[a)]
    \item
        \text{If $\forall_i ~ D_i \leq S_i$ \quad then $Y_{ij} = Y_{ji} = 0, \quad i,j = 1,2; j \neq i$}
    \item
        \text{If $\forall_i ~ D_i \geq S_i$ \quad then $Y_{ij} = Y_{ji} = 0, \quad i,j = 1,2; j \neq i$}
    \item
        \text{If $D_i < S_i$ \quad and $D_j > S_j,$ then}
    \begin{align*}
        \text{$Y_{ij} = \min\{S_i - D_i,D_j-S_j\}$ and $Y_{ji} = 0,\quad i,j = 1,2; j \neq i\,$.}
    \end{align*}
\end{enumerate}





Reference used in spare parts Mario Guajardo: %http://www.din.uem.br/sbpo/sbpo2012/pdf/arq0346.pdf


Reid R.D., Sanders N.R. - Operations Management An Integrated Approach.pdf


* usecase: demand spikes around some warehouses, even if we cut PO, to cover the vendor lead time

* amzn: 1) use a clustering technique for demand (no, use which FC can ship within one day to list of zip codes, then do mapping like that evt også inkluder ship one day + <200km), create demand regions. 2) see which FCs can deliver within 1 day to the regions, map FC<->region. 3) Predict LIS offenders per region. 

Nice overview https://core.ac.uk/download/pdf/41829234.pdf:
. Lateral Transshipments based on Availability (TBA) policy (Banerjee et al., 2003): in literature it is often named as Emergency Lateral Transshipment (ELT). In this case, at time t, it is checked if one or more warehouses are in stock shortage. 

Retailers share inventory, online grocery: https://www.sciencedirect.com/science/article/abs/pii/S0305054821000290?via%3Dihub

Same as above:
https://www.researchgate.net/publication/349066989_Lateral_Inventory_Share-based_Models_for_IoT-Enabled_E-Commerce_Sustainable_Food_Supply_Networks


% SLUTT SKRIV HER
\end{document}