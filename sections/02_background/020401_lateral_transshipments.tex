\documentclass[../../main.tex]{subfiles}
\begin{document}
% START SKRIV HER

\section{Lateral Transshipments}

Lateral transshipments are shipments between warehouses at the same echelon (level) in the supply chain. It can be viewed as another form of inventory pooling, where warehouses of the same echelon share inventory since one warehouse can send a shipment that is needed in another warehouse. 

By using transshipments, one can achieve the benefits of risk pooling without a central warehouse.

Paterson (2011) categorizes lateral transshipments as either proactive or reactive. A proactive transfer, or preventive transfer, is a 

Nice overview https://core.ac.uk/download/pdf/41829234.pdf:
. Lateral Transshipments based on Availability (TBA) policy (Banerjee et al., 2003): in literature it is often named as Emergency Lateral Transshipment (ELT). In this case, at time t, it is checked if one or more warehouses are in stock shortage. 

% SLUTT SKRIV HER
\end{document}