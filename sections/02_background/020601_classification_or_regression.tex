\documentclass[../../main.tex]{subfiles}
\begin{document}
% START SKRIV HER

\section{Classification or Regression}
The initial scope for this thesis was to classify whether an item would go out-of-stock in the week after the prediction. This was motivated by a desire to increase service level in warehouses by introducing effective lateral transshipments. By reducing stockouts, the distance between customers and products would shrink and shorter delivery time is achieved. However, the end product would be incomplete at that point. After estimating which items would experience a stockout at which warehouse, it still remained to decide how many units were needed to avoid a stockout, which warehouses had excess inventory, and how many units each warehouse could share. The initial idea for the transshipment policy was that a heuristic using domain knowledge could solve the problem of how much to transfer. However, after discussing with the business owners, the research question evolved into predicting the unknown parameters needed for a lateral transshipment policy. This made the model more actionable, at the expense of added complexity. The rationale was that even though the more complex model would inherently introduce more uncertainty, it would be offset by a higher business value for the company.

% SLUTT SKRIV HER
\end{document}