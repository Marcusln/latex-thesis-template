\documentclass[../../main.tex]{subfiles}
\begin{document}
% START SKRIV HER

\section{Benchmark Models}

\subsection{Classification}
\subsubsection{Simple Heuristics}
Simple heuristics are a good way of quickly determining whether the effort of using machine learning is worth it. After all, we strive for accurate models, but not for accuracy in itself, but for the business value it brings. Hence, the cost of developing machine learning models needs to surpass the improvement gained from more complex models. For our use case, we use two simple heuristics: A product is predicted to go out-of-stock if it already is out-of-stock at the time of prediction. This might actually be a good measure since if a product is already out-of-stock, it could indicate problems with supply, which depending on lead time can take considerable time to improve.

\subsubsection{Logistic regression}

\subsection{Regression}
* Predict same as last week

The use of multivariate models in this thesis is motivated by the complexity of the supply chain facing the case study company. In a simple supply chain, a good demand forecast would suffice to calculate the stockout quantity. To illustrate, this could be a single-echelon supply chain with a small number of warehouses, distinct products and suppliers. In that case, the supply chain is well-behaved using assumptions from literature. The stockout quantity in period $t_1$ would be the forecasted demand in $t_1$ minus the on-hand inventory in $t_0$. In our case, however, with a complex multi-echelon supply chain with many warehouses and suppliers, a complex ordering policy and lateral transshipments allowed, we hypothesize that multivariate models are more suitable as the aforementioned factors can be modelled. For this reason, linear regression is chosen as a benchmark, as opposed to a commonly used forecasting technique such as ARIMA. Testing this hypothesis would be interesting, but outside the scope of this thesis.

\subsection{Linear Regression}

% SLUTT SKRIV HER
\end{document}