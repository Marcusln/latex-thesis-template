\documentclass[../../main.tex]{subfiles}
\begin{document}
% START SKRIV HER

\section{Benchmark Models}

\subsection{Classification}
* Predict same as last week
* Predict the most common class
* Logistic regression

\subsection{Regression}
* Predict same as last week
* Linear regression

https://h1ros.github.io/posts/drop-highly-correlated-features/

The use of multivariate models in this thesis is motivated by the complexity of the supply chain facing the case study company. In a simple supply chain, a good demand forecast would suffice to calculate the stockout quantity. To illustrate, this could be a single-echelon supply chain with a small number of warehouses, distinct products and suppliers. In that case, the supply chain is well-behaved using assumptions from literature. The stockout quantity in period $t_1$ would be the forecasted demand in $t_1$ minus the on-hand inventory in $t_0$. In our case, however, with a complex multi-echelon supply chain with many warehouses and suppliers, a complex ordering policy and lateral transshipments allowed, we hypothesize that multivariate models are more suitable as the aforementioned factors can be modelled. For this reason, linear regression is chosen as a benchmark, as opposed to a commonly used forecasting technique such as ARIMA. Testing this hypothesis would be interesting, but outside the scope of this thesis.

% SLUTT SKRIV HER
\end{document}