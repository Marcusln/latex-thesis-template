\documentclass[../../main.tex]{subfiles}
\begin{document}
% START SKRIV HER

\section{Choice of Policy}

In a review of xx articles, Paterson et al. (2011) presents a table for key characteristics for classifying literature on lateral transshipments (see appendix). The company used in our model is described by these characteristics in TABLE xx. Compared to the articles reviewed, there is a large deviation between literature and a realistic scenario for a modern online retailer. Consequently, it is not straight-forward to find a framework that is suitable.

First, a majority of the literature is on reactive transshipments, often in spare-parts management for the aviation- or automotive industry. In online retail, a proactive approach is preferred, since inventory needs to be in proximity to the customer before demand is realized (to be able to deliver – or promise to do so – in, say, the next day). Second, models in the literature generally attempts to find the optimal policy while taking the entire supply chain into consideration, including ordering policies. As a result, very restrictive assumptions are used when defining the optimal structure of a transshipment policy. By using these restrictive assumptions, the optimal inventory levels are already given, although this is precisely the problem that needs to be solved. In fact, the optimal point in time for a proactive transshipment is currently not known (Paterson et al., 2011). The complexity of the problem is the reason why several papers suggest a near-optimal approach instead (Meissner and Senicheva, 2018; Axsäter, 2003; Agrawal et al., 2004). It follows that research on lateral transshipments are not easily implemented in practical use cases. Finally, this might explain why there is very little research available using real data. To the best of our knowledge, no case studies on proactive transshipment policies exists in the literature.
 
Recent research suggests that machine learning is a natural candidate to help solve combinatorial optimization problems that are computationally too expensive or mathematically not well defined (Bengio et al., 2020). 

Bengio et al: https://arxiv.org/pdf/1811.06128.pdf

Link to literature:
P(OOS) -> ML model classification
Use upper/lower bounds -> Prediction threshold

comprehensive literature review of xx articles in Paterson (2011@), a  was used to search for a transshipment policy in which we can apply the stockout predictions. In TABLE xx, 

\begin{itemize}
\item Number of items: M
\item Number of echelons: 1 or M; Not relevant other than warehouse proximity to customers important, since the aim is to increase CIV? Hmm. Only interested in the customer-facing echelon as this will impact delivery speed.
\item Number of locations: N
\item Identical locations: yes, because what matters is the item (CIV) and transshipment cost, not the warehouse
\item Unsatisfied demand: Lost sales because Internet
\item Orders: Ignored. Irrelevant because we take a snapshot of the supply chain where ordering is included (continuous vs periodic review)
\item Type of transshipment: Combine reactive / proactive because we do reactive transshipments on a regular basis, ie proactive transfers with a reactive trigger
\item Pooling: Partial, because part of inventory is held back to cover future (next week's) demand
\item Decision making: Central, because we want to maximize customer experience ie delivery speed for all customers in an online marketplace
\item Transshipment cost structure: Per item
\end{itemize}




% SLUTT SKRIV HER
\end{document}