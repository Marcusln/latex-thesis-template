\documentclass[../../main.tex]{subfiles}
\begin{document}
% START SKRIV HER

\section{Machine Learning in Supply Chain}


- Limitations
- Machine Learning in Supply chain
- Sustainability
- How our model differs literature
- Generalization, kan dette hjelpe andre bedrifter

Assumptions:
* Demand is evenly spread out so that if an additional in-stock warehouse always mean that the distance to the customers are reduced
* If the online retailer does not have the item in-stock when a customer visits the website, the sale is lost
* The supply chain is a single-echelon for simplicity, only the customer-facing echelon is relevant.
* Assume ordering as exogenous, ie given and already optimized as much as possible. Ie we are trying to maximize the customer experience out of the inventory we already have in the network. Ie optimize inventory placement for delivery speed, ignoring all other costs/contribition. TA DETTE MED I INNLEDNING.
=> this is a limitation since we will have a ordering system optimizing for one thing and transshipment policy for another, ie no global optimization
* If there is a need for replenishment in the remaining period (next week), it will happen.

* Limitations of the myopic approach
* How the policy will interfer with ordering systems

=> relatere resultater til andres forskning
Nevertheless, the proposed policy in this thesis is based on Lee et al. (2007). It is worth noting that other models could have been chosen. By definition, literature share the key concept


Link to literature:
P(OOS) -> ML model classification
Use upper/lower bounds -> Prediction threshold


* Link to SLRP Lee et al. (2007)?
  * Combine proactive and reactive as mentioned in Paterson (2011), use SLRP from Lee (2007).

1.	Predict stockout on a weekly basis (proactive)
2.	If stockout is likely in a warehouse (reactive), take inventory from a warehouse with (the most) surplus inventory to the stockout warehouse
3.	Transship as much as possible of the excess inventory up until the optimal inventory level

*	ABC, ie one stragegy for A = fast items, B = slow, C = ultraslow etc, from risk pooling ref book? 
  * Fast: Optimize for recall
  * Slow: Optimize for precision

\section{Transshipments' importance in delivery speed in Online Retail}
https://en.wikipedia.org/wiki/Kozmo.com

\section{Value of the model}

\section{Importance of IT}
Sophisticated system that is aware of all inventory positions, and linked to the delivery promise offered on the website. If the system knows that inventory is sufficiently close to the customer

\section{Sustainability}

\section{Limitations and Further Research}

% SLUTT SKRIV HER
\end{document}